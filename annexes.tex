\begin{appendices}

\chapter*{Annexe A}
	\addcontentsline{toc}{chapter}{Annexe A}		
	
	Code MATLAB qui calcule la matrice de commandabilité,\label{Annexe A} \hyperref[Co]{Retour vers section 1.1}
	
	\begin{lstlisting}	
S=0.0154;
Sn=5*10^-5;
g=9.81;
H10=0.27474;
H20=0.0299;
H30=0.1368;
H00=0;
a13=0.4753*Sn*sqrt(2*g);
a32=0.4833*Sn*sqrt(2*g);
a20=0.9142*Sn*sqrt(2*g);

R13=(2*sqrt(H10-H30))/a13;
R32=(2*sqrt(H30-H20))/a32;
R20=(2*sqrt(H20-H00))/a20;

A=[-1/(S*R13) 1/(S*R13) 0;
    1/(S*R13) -(1/S)*((1/R13)+(1/R32)) 1/(S*R32);
    0 1/(S*R32) -(1/S)*((1/R32)+(1/R20))]
B=[ 1/S; 0; 0]
C=[1 0 0];
D=[0];

sys=ss(A,B,C,D);

Co=ctrb(sys);
rang_co=rank(Co);
	\end{lstlisting}

\end{appendices}


\chapter*{Annexe B}
	\addcontentsline{toc}{chapter}{Annexe B}		
	
	Code MATLAB qui calcule le gain matriciel K,\label{Annexe B} \hyperref[K]{Retour vers section 1.2}
	
	\begin{lstlisting}	
P2=[-0.12 -0.15];
P=[-0.03333 P2];
K=acker(A,B,P); 
	\end{lstlisting}	
	
\chapter*{Annexe C}
	\addcontentsline{toc}{chapter}{Annexe C}		
	
	Code MATLAB qui calcule la valeur du pré-compensateur N,\label{Annexe C} \hyperref[N]{Retour vers section 1.2}
	
	\begin{lstlisting}	
Abf=A-B*K;
N=1/(C*inv(-Abf)*B);
	\end{lstlisting}		
	
\chapter*{Annexe D}
	\addcontentsline{toc}{chapter}{Annexe D}		
	
	Code MATLAB qui calcule la matrice d'observabilité,\label{Annexe D} \hyperref[OBSV]{Retour vers section 2.1.1}
	
	\begin{lstlisting}	
OBSV=obsv(sys);
rang_OBSV=rank(OBSV);
	\end{lstlisting}		
	
	
	\chapter*{Annexe E}
	\addcontentsline{toc}{chapter}{Annexe E}		
	
	Code MATLAB qui calcule les élements de l'observateur,\label{Annexe E} \hyperref[calobs]{Retour vers section 2.1.2}
	
	\begin{lstlisting}	
B1= B(1);
B2= [B(2);B(3)];

A11= A(1,1);
A21= [A(2,1);A(3,1)];
A22= [A(2,2) A(2,3);A(3,2) A(3,3)]; 
A12= [A(1,2) A(1,3)];   

G= acker(A22',A12',[5*P2])';

F= A22-(G*A12);

Gtilde= F*G-G*A11+A21;
   
Htilde= B2-G*B1;
	\end{lstlisting}	
