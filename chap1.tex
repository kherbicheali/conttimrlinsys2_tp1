 \chapter{\textsc {Analyse et calcul d'une loi de commande par retour d'état} }
 %\chaptermark{\textsc {Analyse et calcul d'une loi de commande par retour d'état}}
 
	\section{\textsc {Commandabilité du modèle linéarisé}} 
 	
 	\paragraph{}
 		Calcul de la matrice de commandabilité $Co=\begin{bmatrix} B&BA&A^{2}B \end{bmatrix} $:
 		
 		\begin{center}
			
			$Co$=$\begin{bmatrix}
			64.9351&-0.5975&0.0110\\
			0&0.5975&-0.0173\\
			0&0&0.0064
			\end{bmatrix}$	
			 			
		\end{center} 		 
		
		La matrice de commandabilité est triangulaire supérieure ce qui fait que son rang vaut 3 car il n'éxiste aucune relation linéaire entre ses colonnes ou entre ses lignes.\\
		\textbf{Conclusion:} Le modèle linéarisé est bien commandable vu que le rang de la matrice $Co$ est égale au nombre de valeurs propres que possède la matrice dynamique $A$. 
		
		\section{\textsc {La loi de commande par retour d'état}} 
		
		\begin{center}
		%\includegraphics[scale=0.4]{bobb.png} 
		\captionof{figure}{\textit Schéma SIMULINK du système en boucle fermée avec retour d'état}
		\label{fig1}
		\end{center}
		
		\paragraph{}
			Afin de calculer les paramètres du retour d'état soit le gain matriciel $K$ et le pré-compensateur $N$ nous devons choisir des valeurs pour les pôles désirés $P_{des}=\begin{pmatrix} P_1 & P_2 & P_3 \end{pmatrix} $ qui respectent les spécificités du cahier des charges.\\
		
		\paragraph{} Du cours de SLI2 on trouve que le temps de réponse $t_r =\frac{3}{|Re(vp)|} $ avec $vp$ : valeur propre et vu qu'une spécificité dit que $t_r \leqslant 90s$ alors on trouve:\\
		
		\begin{center}
				
				$t_r =\frac{3}{|Re(vp)|}\leqslant 90 $\\[1cm]
				$\Rightarrow|Re(vp)| \geqslant \frac{3}{90} \simeq 0.033 $\\[1cm]
				$\Rightarrow Re(vp) \geqslant 0.033 \wedge Re(vp) \leqslant -0.033$
		\end{center}
		Si $\Rightarrow Re(vp) \geqslant 0.033$ alors notre système est instable, donc la plus lente valeur propre autrement appelée mode dominant doit être inférieur ou égal à $-0.033$.\\\\
		Nous choisissons alors les valeurs désirés suivantes $P_{des}=\begin{bmatrix} -0.033&-0.15&-0.12 \end{bmatrix} $